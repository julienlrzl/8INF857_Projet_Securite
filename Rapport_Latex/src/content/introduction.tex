\begin{introduction}

Dans ce contexte, ce projet a pour objectif la conception et le déploiement d’un
\textit{système de détection d’anomalies et de gestion de logs}. 
L’approche consiste à mettre en place une chaîne complète allant de la collecte des journaux
jusqu’à leur visualisation et leur analyse via une interface conviviale. 
L’architecture retenue repose sur quatre briques logicielles complémentaires :
\begin{itemize}
    \item \textbf{Suricata}, un système de détection d’intrusions (IDS/IPS) chargé
    d’analyser le trafic réseau et de générer des alertes en temps réel ;
    \item \textbf{syslog-ng}, utilisé pour centraliser les journaux du système et des applications ;
    \item \textbf{Elasticsearch}, base de données NoSQL permettant l’indexation et la recherche rapide
    des événements collectés ;
    \item \textbf{Kibana}, une interface web offrant des fonctionnalités de visualisation et de
    création de tableaux de bord.
\end{itemize}

Le projet doit également inclure l’implémentation de plusieurs scénarios d’attaque
simulés. Ces cas d’intrusion permettront de valider la capacité du système à détecter
différents comportements malveillants et à générer des alertes exploitables par
l’administrateur.

Ce rapport présente dans un premier temps l’environnement mis en place et les choix
technologiques retenus. Il détaille ensuite la configuration et l’intégration des outils,
avant de décrire et d’analyser cinq scénarios d’intrusion représentatifs. 
Enfin, une partie est consacrée à la visualisation des résultats, à l’évaluation des limites
du système et aux perspectives d’amélioration.

\end{introduction}
