\chapter{Mise en place de l’environnement}
\label{chap:chap_1}


\section{Environnement de travail}
Le projet a été réalisé par une équipe de quatre personnes. Chaque membre a travaillé sur une machine hôte différente (Windows ou macOS), mais l’ensemble du projet a été uniformisé à travers l’utilisation d’une machine virtuelle Ubuntu. Ce choix garantit un environnement cohérent, reproductible et isolé, permettant de tester des scénarios d’attaques sans risque pour les machines personnelles.

L’environnement retenu est le suivant :
\begin{itemize}
    \item \textbf{Système d’exploitation invité} : Ubuntu 22.04 LTS (Linux)
    \item \textbf{Hyperviseurs utilisés} : VMware Fusion (macOS) et VMware Workstation/VirtualBox (Windows)
    \item \textbf{Ressources allouées} : 2 vCPU, 4 Go de mémoire vive, 40 Go de disque
    \item \textbf{Interface réseau} : \texttt{ens160}, configurée en mode NAT
\end{itemize}

Ce choix d’architecture permet de travailler de manière collaborative tout en garantissant que les configurations, les scripts et les fichiers produits sont compatibles sur toutes les machines de l’équipe.

\section{Briques logicielles nécessaires}
Le projet repose sur quatre composants principaux :
\begin{itemize}
    \item \textbf{Suricata} : un système de détection et de prévention d’intrusions (IDS/IPS), chargé d’analyser le trafic réseau et de générer des alertes.
    \item \textbf{syslog-ng} : un collecteur de logs, utilisé pour centraliser les journaux générés par le système et par Suricata.
    \item \textbf{Elasticsearch} : une base de données orientée recherche, permettant d’indexer et de stocker les logs collectés.
    \item \textbf{Kibana} : une interface web de visualisation connectée à Elasticsearch, permettant d’explorer et d’analyser les logs.
\end{itemize}

\section{Installation des composants}
L’installation a été réalisée en ligne de commande sur Ubuntu. Les principales étapes sont résumées ci-dessous.

\subsection*{Suricata}
\begin{verbatim}
sudo apt install -y suricata suricata-update
sudo suricata-update
sudo systemctl restart suricata
\end{verbatim}
La commande \texttt{suricata --build-info} permet de vérifier que l’installation est correcte.

\subsection*{syslog-ng}
\begin{verbatim}
sudo apt install -y syslog-ng
systemctl status syslog-ng
\end{verbatim}

\subsection*{Elasticsearch}
Téléchargement et extraction de la version ARM64 :
\begin{lstlisting}[style=bashstyle]
curl -LO https://artifacts.elastic.co/downloads/elasticsearch/elasticsearch-8.15.3-linux-aarch64.tar.gz
tar -xzf elasticsearch-8.15.3-linux-aarch64.tar.gz
cd elasticsearch-8.15.3
./bin/elasticsearch -E discovery.type=single-node -E xpack.security.enabled=false
\end{lstlisting}

\subsection*{Kibana}
Téléchargement et extraction :
\begin{lstlisting}[style=bashstyle]
curl -LO https://artifacts.elastic.co/downloads/kibana/kibana-8.15.3-linux-aarch64.tar.gz
tar -xzf kibana-8.15.3-linux-aarch64.tar.gz
cd kibana-8.15.3
./bin/kibana
\end{lstlisting}

\section{Automatisation par alias}
Afin de simplifier le lancement des différents composants, des alias ont été définis dans le fichier \texttt{~/.bashrc}. 
Ces raccourcis permettent à l’équipe de démarrer ou d’arrêter les services (Elasticsearch, Kibana, Suricata) avec des commandes simples, 
plutôt que de retaper à chaque fois des lignes longues et complexes. 
Cette approche améliore la lisibilité, réduit les erreurs de saisie et accélère les tests.